\documentclass{article}

\usepackage[T2A]{fontenc}
\usepackage[utf8]{inputenc}
\usepackage[russian]{babel}

\title{Компьютерная графика}
\author{Лисид Лаконский}
\date{December 2022}

\usepackage{multienum}
\newtheorem{definition}{Определение}

\begin{document}

\maketitle
\tableofcontents
\pagebreak

\section{Компьютерная графика - 05.12.2022}

\subsection{Проецирование}

\subsubsection{Методы проецирования}

\begin{flushleft}

\begin{definition}

Проецирование — процесс получения проекций предмета на какой-либо поверхности (плоской, цилиндрической, сферической, конической) с помощью проецирующих лучей.

\end{definition}

\hfill

Метод проекций предполагает наличие \textbf{плоскости проекций}, \textbf{объекта проецирования} и \textbf{проецирующих лучей}.

\begin{definition}

Центральное проецирование — это общий случай проецирования геометрических объектов. Проецирование осуществляется из точки S — центра проецирования на плоскость P — плоскость проекций.

\end{definition}

\begin{definition}

Параллельное проецирование — частный случай центрального проецирования, при котором все проецирующие лучи между собой параллельны и центр проецирования удален в бесконечность.

\end{definition}

\begin{definition}

Косоугольное проецирование — проецирующие лучи составляют с плоскостью проекций угол, не равный $90^{\circ}$.

\end{definition}

\begin{definition}

Прямоугольное (ортогональное) проецирование — проецирующие лучи перпендикулярны плоскости проекций.

\end{definition}

\subsubsection{Проецирование точки}

\paragraph{Основные правила ортогонального проецирования точки}

\begin{enumerate}
    \item Положение точки в пространстве определяется тремя координатами: $A(x, y, z)$
    \item Положение точки на плоскости определяется двумя координатами: $a(x, y);$ $a'(x, z)$, $a''(y, z)$
    \item Две проекции точки определяют положение ее третьей проекции; две проекции точки определяют ее положение в пространстве.
    \item Две проекции находятся на одном перпендикуляре (линии связи) к оси проекций, их разделяющей.
\end{enumerate}

\subsubsection{Проецирование прямой}

\begin{definition}
    Линия — это множество всех последовательных положений двигающейся точки.
\end{definition}

\begin{definition}
    Прямая линия — линия, образованная движением точки, не меняющей своего направления
\end{definition}

\paragraph{Положение прямой в пространстве}

Относительно плоскостей проекции прямая может занимать различные положения:

\begin{enumerate}
    \item не параллельное ни одной из плоскостей проекций
    \item параллельные одной из плоскостей проекций
    \item параллельное двум плоскостям проекций, то есть перпендикулярное третьей
\end{enumerate}

\begin{definition}
    Прямая общего положения — прямая, не параллельная ни одной из плоскостей проекций.
\end{definition}

\begin{definition}
    Прямые частного положения — прямые, параллельные или перпендикулярные плоскости проекций
\end{definition}

Прямые частного положения можно разделить на:

\begin{enumerate}
    \item прямые, параллельные плоскости проекций — прямые уровня
    \item прямые, перпендикулярные плоскости проекций — проецирующие прямые
\end{enumerate}

\begin{definition}
    Фронтальная прямая — прямая, параллельная фронтальной плоскости проекций $V$.
\end{definition}

\begin{definition}
    Профильная прямая — прямая, параллельная профильной плоскости проекций $W$.
\end{definition}

Существуют три вида \textbf{проецирующих прямых}:

\begin{enumerate}
    \item Прямая перпендикулярная горизонтальной плоскости проекций – \textbf{горизонтально-проецирующая прямая}.
    \item Прямая, перпендикулярная фронтальной плоскости проекций – \textbf{фронтально-проецирующая прямая}.
    \item Прямая, перпендикулярная профильной плоскости проекций – \textbf{профильно-проецирующая прямая}.
\end{enumerate}

\paragraph{Следы прямой}

\begin{definition}
    Точки пересечения прямой линии с плоскостями проекций называются следами прямой.
\end{definition}

\paragraph{Взаимное положение двух прямых в пространстве}

Прямые в пространстве могут занимать следующие взаимные положения:

\begin{enumerate}
    \item пересекаться, то есть иметь одну общую точку
    \item скрещиваться, то есть не иметь общей точки
    \item быть параллельными, когда точка пересечения прямых удалена в бесконечность
\end{enumerate}

\begin{definition}
    Конкурирующими точками называются точки, лежащие на одной линии связи, но на разных прямых
\end{definition}

\subsubsection{Проецирование плоскости}

На чертеже плоскость может быть задана проекциями:

\begin{multienumerate}
    \mitemxxx{трех точек, не лежащих на одной прямой}{прямой и точки, не лежащей на этой прямой}{двух пересекающихся прямых}
    \mitemxxx{двух параллельных прямых}{любой плоской фигуры}{следами плоскости}
\end{multienumerate}

\paragraph{Следы плоскости}

\begin{definition}
    Следами плоскости называется линия пересечения плоскости с плоскостью проекций.
\end{definition}

\paragraph{Положение плоскости в пространстве}

\begin{definition}
    Плоскость общего положения — плоскость, не перпендикулярная ни к одной из плоскостей проекций, называют плоскостью общего положения.
\end{definition}

Плоскости уровня:

\begin{multienumerate}
    \mitemxxx{фронтальная}{горизонтальная}{профильная}
\end{multienumerate}

\subsubsection{Взаимное положение точек, прямых и плоскостей}

Точка \textbf{принадлежит плоскости}, если она принадлежит прямой, принадлежащей плоскости.

Прямая \textbf{принадлежит плоскости}, если она проходит через две точки, принадлежащие данной плоскости.

\end{flushleft}

\end{document}