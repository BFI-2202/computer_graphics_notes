\documentclass{article}

\usepackage[T2A]{fontenc}
\usepackage[utf8]{inputenc}
\usepackage[russian]{babel}

\title{Компьютерная графика}
\author{Лисид Лаконский}
\date{October 2022}

\begin{document}

\maketitle
\tableofcontents
\pagebreak

\section{Компьютерная графика - 10.10.2022}

Косяк предыдущей работы - плохое оформление списка литературы - неполная запись про ГОСТ.

\subsection{Векторная графика}

\begin{flushleft}

Векторная графика - это категория компьютерной графики, которая для представления изображения использует фигуры, построенные по математическим формулам.

Преимущества векторной графики:

\begin{enumerate}
    \item Масштабирование без потери качества
    \item Масштабирование изображения без увеличения объема памяти
    \item Масштабирование без ущерба четкости, резкости, цвету
    \item Изображения в векторном формате легко конвертируется в растровый формат
\end{enumerate}

Недостатки векторной графики:

\begin{enumerate}
    \item Сложность создания изображений
    \item Плохое качество конвертирование из растровой графики в векторную
\end{enumerate}

\end{flushleft}

\end{document}
