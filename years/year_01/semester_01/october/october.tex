\documentclass{article}

\usepackage[T2A]{fontenc}
\usepackage[utf8]{inputenc}
\usepackage[russian]{babel}

\title{Компьютерная графика}
\author{Лисид Лаконский}
\date{October 2022}

\begin{document}

\maketitle
\tableofcontents
\pagebreak

\section{Компьютерная графика - 10.10.2022}

Косяк предыдущей работы - плохое оформление списка литературы - неполная запись про ГОСТ.

\subsection{Векторная графика}

\begin{flushleft}

Векторная графика - это категория компьютерной графики, которая для представления изображения использует фигуры, построенные по математическим формулам.

Преимущества векторной графики:

\begin{enumerate}
    \item Масштабирование без потери качества
    \item Масштабирование изображения без увеличения объема памяти
    \item Масштабирование без ущерба четкости, резкости, цвету
    \item Изображения в векторном формате легко конвертируется в растровый формат
\end{enumerate}

Недостатки векторной графики:

\begin{enumerate}
    \item Сложность создания изображений
    \item Плохое качество конвертирование из растровой графики в векторную
\end{enumerate}

\end{flushleft}

\pagebreak
\section{Компьютерная графика - 24.10.2022}

\subsection{Трехмерная графика}

\begin{flushleft}

\textbf{Трехмерная графика} - раздел компьютерной графики, посвященный методам создания изображений или видео путем моделирования объемных объектов в трехмерном пространстве.

\hfill

Алгоритм получения трехмерного изображения:

\begin{enumerate}
    \item Моделирование - создание трехмерной модели
    \item Рендеринг - построение проекции
    \item Вывод изображений на устройство вывода - дисплей или принтер
\end{enumerate}

\textbf{Задача трехмерного моделирования} - описание и размещение трехмерных моделей на сцене с помощью геометрических преобразований в соответствии с требованиями к будущему изображению.

\hfill

Для пространственного моделирования объекта требуется:

\begin{enumerate}
    \item Спроектировать и создать виртуальный каркас объекта
    \item Спроектировать и создать виртуальные материалы
    \item Присвоить материалы различным частям поверхности объекта
    \item Настроить физические параметры пространства, в котором будет действовать объект
    \item Задать траектории движения объектов
    \item Рассчитать результирующую последовательность кадров
    \item Наложить поверхностные эффекты на итоговый анимационный ролик
\end{enumerate}

\textbf{Сцена - виртуальное пространство моделирования}

Сцена включает в себя несколько категорий объектов:

\begin{multienumerate}
    \mitemxx{Геометрия}{Материалы}
    \mitemxx{Источники света}{Виртуальные камеры}
    \mitemxx{Силы и взаимодействия}{Дополнительные эффекты}
\end{multienumerate}

\textbf{Рендеринг} - процесс получения изображения по модели с помощью компьютерной программы.

\hfill

\textbf{Компьютерная анимация} - вид анимации, создаваемый при помощи компьютера.

\textbf{Мультимедиа} - это объединение высококачественного изображения на экране компьютера со звуковым сопровождением.

Под \textbf{компьютерной анимацией} понимают получение движущихся изображений на экране дисплея.

\hfill

\textbf{Направления применения анимации}:

\begin{multienumerate}
    \mitemxx{Анимационные заставки}{Анимация или анимационные вставки}
    \mitemxx{Презентационный ролик}{Анимация открытки}
    \begin{center}
        И так далее...
    \end{center}
\end{multienumerate}

\textbf{Перечень возможностей Blender}:

\begin{multienumerate}
    \mitemxx{3D-моделирование}{Анимация}
    \mitemxx{Эффекты}{Опции рисования}
\end{multienumerate}

\subsection{Электрические схемы}

\textbf{Конструкторская графика} - это инструмент создания технических изделий.

\hfill

\textbf{Система автоматизированного проектирования} (САПР) - автоматизированная система, реализующая информационную технологию выполнения функций проектирования, предоставляющая собой организационно-техническую систему, предназначенную для автоматизации процесса проектирования, состоящую из персонала и комплекса технических программных и других средств автоматизации его деятельности.

\hfill

\textbf{Конструкторский документ} - это документ, который в отдельности или в совокупности с другими документами определяет конструкцию изделия и имеет содержательную и реквизитную части, в том числе установленные подписи.

\textbf{Конструкторская документация} - это совокупность конструкторских документов, содержащих данные, необходимые для проектирования, изготовления, контроля, приемки, поставки, эксплуатации, ремонта, модернизации, утилизации изделия.

\hfill

\textbf{Графический документ} - это конструкторский документ, содержащий в основном графическое изображение изделия и/ли его составных частей, отражающее взаимное расположение и функционирование этих частей, их внутренние и внешние связи

\hfill

Правила разработки и оформления конструкторской документации устанавливаются государственными стандартами \textbf{Единой системы конструкторской документации}.

\textbf{ЕСКД} - комплекс стандартов, устанавливающих взаимовязанные правила, требования и нормы по разработке, оформлению и обращению конструкторской документации

\hfill

\textbf{Техническая документация} - набор документов, используемых при проектировании (конструировании), изготовлении и использовании объектов техники: зданий, сооружений, промышленных изделий, включая программное и аппаратное обеспечение.

В составе технической документации выделяют:

\begin{enumerate}
    \item Конструкторские документы, включая чертежи, спецификация, пояснительные записки, технические отчеты, технические условия, эксплуатационные и ремонтные документы
    \item Технологические документы, включая документы, необходимые для организации производства и ремонта изделия
    \item Программные документы, сопровождающие программы для электронно-вычислительных машин
\end{enumerate}

Конструкторские документы подразделяются на текстовые и графические.

\end{flushleft}

\end{document}
