\documentclass{article}

\usepackage[T2A]{fontenc}
\usepackage[utf8]{inputenc}
\usepackage[russian]{babel}

\title{Компьютерная графика}
\author{Лисид Лаконский}
\date{November 2022}

\usepackage{multienum}

\begin{document}

\maketitle
\tableofcontents
\pagebreak

\section{Компьютерная графика - 21.11.2022}

\subsection{Стандарты ЕСКД. Правила оформления чертежей}

\subsubsection{Список основных стандартов}

\begin{flushleft}

\begin{multienumerate}
    \mitemxxx{Форматы, рамка чертежа (ГОСТ 2.301-68)}{Основная надпись чертежа (ГОСТ 2.104-2006)}{Масштабы изображений (ГОСТ 2.302-68)}
    \mitemxxx{Линии, применяемые на чертежах (ГОСТ 2.303-68)}{Шрифт чертежный (ГОСТ 2.304-68)}{Нанесение размеров (ГОСТ 2.306-68)}
\end{multienumerate}

\subsubsection{Стандарты ЕСКД}

\textbf{Стандарты ЕСКД} - это документы, которые устанавливают единые правила выполнения и оформления конструкторских документов во всех отраслях промышленности, строительства, транспорта.

\subsubsection{Масштабы}

\textbf{Масштаб чертежа} - отношение линейных размеров изображения предмета на чертеже к его соотвествующим действительным размерам.

Согласно ГОСТ 2.302-68, масштабы должны выбираться из следующих рядов:

\begin{enumerate}
    \item Натуральная величина - 1:1
    \item Масштабы уменьшения - 1:2; 1:2.5; 1:4; 1:10; 1:20 и так далее...
    \item Масштабы увеличения
\end{enumerate}

\end{flushleft}

\subsubsection{Линии, применяемые на чертежах}

За исходную принята сплошная толстая основная линия. Толщина основной линии обозначается как $S$ и может быть выбрана в пределах от 0.5 до 1.4 мм

\subsubsection{Шрифты чертежные}

ГОСТ 2.304-81 установлены следующие размеры шрифтов: 1,8; 2,5; 3,5; 5; 7; 10; 14; 20 и так далее...

Стандартом установлены следующие типы шрифтов:

\begin{multienumerate}
    \mitemxx{Тип А без наклона}{Тип А с наклоном около $75^\circ$}
    \mitemxx{Тип Б без наклона}{Тип Б с наклоном около $75^\circ$}
\end{multienumerate}

\end{document}