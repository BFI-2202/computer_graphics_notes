\documentclass{article}

\usepackage[T2A]{fontenc}
\usepackage[utf8]{inputenc}
\usepackage[russian]{babel}

\usepackage{multienum}

\title{Компьютерная графика}
\author{Лисид Лаконский}
\date{November 2022}

\begin{document}

\maketitle
\tableofcontents
\pagebreak

\section{Компьютерная графика - 07.11.2022}

\subsection{Схемы электрические}

\begin{flushleft}

\textbf{Электрическая схема} - это документ, составленный в виде условных изображений или обозначений составных частей изделия, действующих при помощи электрической энергии и их взаимосвязей. Электрические схемы являются разновидностью схем изделия и обозначаются в шифре основной надписи буквой Э.

Правила выполнения всех типов электрических схем установлены ГОСТ 2.702-2011, при выполнении схем цифровой вычислительной техники руководствуются ГОСТ 2.708-81.

\subsubsection{Классификация составных частей РЭА}

Классификацию составных частей РЭА (радио-электронная аппаратура) определяет ГОСТ 2.701-2008.

\subsubsection{Типы электрических схем}

\textbf{Схема} - документ, на котором показаны в виде условных изображений или обозначений составные части изделия и связи между ними.

\hfill

Типы электрических схем также определяются ГОСТ 2.701-2008:

\begin{multienumerate}
    \mitemxxx{Структурная}{Функциональная}{Принципиальная}
    \mitemxxx{Соединений}{Подключения}{Общая}
    \mitemxx{Расположения}{Объединенная (код - 0)}
\end{multienumerate}

\paragraph{Схемы электрические структурные: определение и алгоритм посроения}

Согласно определению (ГОСТ 2.701-84):

\textbf{Схема электрическая структурная} - это графический конструкторский документ, на котором показаны основные функциональные части изделия, их назначение и взаимосвязи

\textbf{Алгоритм построения:} построение чертежа - выписка названий элементов - соединение элементов линиями со стрелками - оформление документа - заполнение основной надписи.

\paragraph{Схемы электрические функциональные}

Согласно определению (ГОСТ 2.701-84):

\textbf{Схема электрическая функциональная} - это графический конструкторский документ, который изъясняет определенные процессы, протекающие протекающие в отдельных функциональных цепях изделия или в изделии в целом

\paragraph{Схемы электрические принципиальные}

Согласно определению (ГОСТ 2.701-84):

\textbf{Схема электрическая принципиальная} - это графический конструкторский документ, который определяет полный состав элементов и связей между ними и дает детальное представление о принципах работы изделия.

\subsubsection{Графические и позиционные обозначения}

\textbf{Условное графическое обозначение} - одна из составных частей описания компонента. УГО используется для обозначения компонента на схемах.

УГО делятся на две катеории:

\begin{enumerate}
    \item Типовые УГО, заданные в стандартах системы
    \item Специфические УГО, созданные в рамках компонента
\end{enumerate}

ГОСТ, определяющий УГО радиоэлектронных устройств: ГОСТ 2.737-68

\subsubsection{Стандартные позиционные обозначения и маркировка устройств и элементов}

Буквенные обозначения электронных компонентов на отечественных схемах регламентированы ГОСТ 2.710-81 "Обозначения буквенно-цифровые в электрических схемах"

\end{flushleft}

\end{document}