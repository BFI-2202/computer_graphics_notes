\documentclass{article}

\usepackage[T2A]{fontenc}
\usepackage[utf8]{inputenc}
\usepackage[russian]{babel}

\usepackage{multienum}

\title{Компьютерная графика}
\author{Лисид Лаконский}
\date{November 2022}

\begin{document}

\maketitle
\tableofcontents
\pagebreak

\section{Компьютерная графика - 09.11.2022}

\subsection{Практическая работа №4, часть 1}

\subsubsection{Контрольные вопросы}

\begin{enumerate}
    \item Понятие "векторная графика": способ представления графических объектов и изображений (формат описания), основанный на математическом описании элементарных геометрических объектов
    \item Перечислите компоненты интерфейса программы Inkscape: строка меню; панель инструментов; контекстная панель инструментов; разметка, линейки, направляющие и сетки; окно инструментов; рабочая область; палитра; строка состояния
    \item Перечислите простейшие геометрические объеты программы: прямоугольники, эллипсы, многоугольники, спирали
    \item Сформулируйте понятие "заливка": заливка — это цвет, узор, текстура, рисунок или градиент, примененные к внутренней части фигуры
    \item Перечислите виды заливок и дайте им определения: сплошной цвет - заливка одним цветом; линейный градиент - заливка несколькими цветами с плавными переходами от одного цвета к другому; радиальный градиент - аналогично линейному градиенту, но один цвет переходит в другой не вдоль прямой линии, а словно круги по воде вокруг точки; текстура; образец
    \item Расписать алгоритм выполнения заливки: выбор инструмента заливки - настройка параметров инструмента заливки - применение инструмента заливки на желаемую область
    \item Перечислите типы узлов и дайте им определения: это какая-то шиза, по-моему, узел есть узел, и у него нет никаких типов
    \item Перечислите инструменты редактирования форм кривых: инструмент редактирования узлов контура или рычагов узла позволяет добавлять узлы, удалять узлы, объединять узлы, разбивать узлы, делать их острыми, прямыми, сглаженными, симметричными и так далее
    \item Алгоритм редактирования форм объекта: преобразовать объект в контур; выбрать инструмент редактирования узлов контура; редактировать узлы контура так, как пожелается
    \item Операции над объектами, перечислить и пояснить алгоритм выполнения: копирование - правка - продублировать; выравнивание - объект - выровнять и расставить; порядок расположения - через меню объект; группировка - выделить объекты - объект - сгруппировать; логические операции можно выполнять через меню контур - объекты должны быть контурами - если не контуры, то нужно преобразовать в контуры
\end{enumerate}

\subsection{Практическая работа №4, часть 2}

\subsubsection{Контрольные вопросы}

\begin{enumerate}
    \item Особенности работы с инструментом «текст»: в отличие от работы с текстом в растровых графических редакторах, в векторных графических редакторах нет необходимости в создании нового слоя при добавлении текста; выбор цвета текста осуществляется с помощью палитры; выбор настроек шрифта происходит с помощью контекстной панели инструментов; панель работы с текстом можно вызвать нажав shift+ctrl+T; текст возможно размещать по контуру и верстать в какой-либо блок
    \item Достоинства векторной графики: малый объем занимаемой памяти; масштабирование без потери качества, увеличения объема памяти, ущерба четкости, резкости цвету; возможность лёгкого конвертирования в растровый формат (из лекции)
    \item Недостатки векторной графики: сложность создания изображений, плохое качество конвертирования из растровой графики в векторную (из лекции)
    \item Форматы файлов векторной графики: svg, eps, ai (Adobe Illustrator), cdr (Corel Draw) и так далее; кроме того, pdf может содержать в себе векторную графику. Inkscape поддерживает все эти форматы, основной формат - svg
    \item Алгоритм выполнения операции «разрезание контура и текста»: сначала необходимо оконтурить объект (текст), если контур — он уже контур, его не надо оконтуривать ещё; для текста: далее выделить все — правой кнопкой — разгруппировать. Потом для всего: инструментом «ластик» (или хз, как он правильно называется) разрезать, далее контур — разбить.
\end{enumerate}
\end{document}